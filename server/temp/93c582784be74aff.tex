

\documentclass{article}
\title{Business vs. Startup}
\author{Your Name}
\date{\today}

\begin{document}

\maketitle

\section{Introduction}

This assignment explores the key differences between established businesses and startups.  We will examine their contrasting characteristics, focusing on aspects such as size, goals, funding, risk tolerance, and management structure.

\section{Business Characteristics}

Established businesses typically possess the following attributes:

\begin{itemize}
    \item \textbf{Size:} Larger scale operations with established market share.
    \item \textbf{Goals:} Focus on maintaining market position, profitability, and steady growth.
    \item \textbf{Funding:}  Often rely on established revenue streams and potentially debt financing.
    \item \textbf{Risk Tolerance:}  Generally risk-averse, prioritizing stability and predictability.
    \item \textbf{Management:} Formalized hierarchical structures with established processes and procedures.
\end{itemize}

\section{Startup Characteristics}

Startups, on the other hand, are characterized by:

\begin{itemize}
    \item \textbf{Size:} Smaller, often nascent operations with limited market share.
    \item \textbf{Goals:} Focused on rapid growth, innovation, and market disruption.
    \item \textbf{Funding:}  Frequently rely on venture capital, angel investors, or bootstrapping.
    \item \textbf{Risk Tolerance:}  High risk tolerance, embracing experimentation and innovation.
    \item \textbf{Management:} Often less formal structures, with agile and adaptable teams.
\end{itemize}

\section{Key Differences}

A table summarizing the key differences:

\begin{tabular}{|l|l|l|}
\hline
Feature & Business & Startup \\
\hline
Size & Large & Small \\
Goals & Stability, Profitability & Growth, Disruption \\
Funding & Revenue, Debt & VC, Angel Investors \\
Risk Tolerance & Low & High \\
Management & Hierarchical & Agile \\
\hline
\end{tabular}


\section{Conclusion}

While both businesses and startups aim for success, their approaches, strategies, and risk profiles differ significantly. Understanding these differences is crucial for entrepreneurs, investors, and anyone navigating the business world.


\end{document}
