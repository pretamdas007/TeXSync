

\documentclass{article}
\usepackage[utf8]{inputenc}
\usepackage{amsmath}
\usepackage{graphicx}
\usepackage{natbib}
\usepackage{hyperref}

\title{A Research Paper Title}
\author{Your Name(s) \\ Affiliation(s)}
\date{\today}

\begin{document}

\maketitle

\begin{abstract}
This is the abstract of the research paper.  It should concisely summarize the main points, including the research question, methods, and findings.
\end{abstract}

\section{Introduction}
This section introduces the research topic, provides background information, states the research question(s) or hypothesis(es), and outlines the paper's structure.  It should clearly explain the significance and relevance of the research.

\section{Literature Review}
This section reviews existing literature relevant to the research topic. It should critically evaluate previous work and identify gaps in knowledge that the current research addresses.

\section{Methodology}
This section describes the research design, data collection methods, and data analysis techniques used in the study. It should be detailed enough for other researchers to replicate the study.

\subsection{Participants}
Describe the participants in the study, including their characteristics (e.g., age, gender, demographics).

\subsection{Materials}
Describe any materials used in the study (e.g., questionnaires, experimental stimuli).

\subsection{Procedure}
Describe the steps involved in conducting the study.

\subsection{Data Analysis}
Describe the statistical methods or techniques used to analyze the data.

\section{Results}
This section presents the findings of the study in a clear and concise manner. Use tables and figures to present the data effectively.  Avoid interpreting the results in this section; that is done in the discussion.

\section{Discussion}
This section interprets the results of the study in relation to the research question(s) or hypothesis(es). It discusses the implications of the findings, limitations of the study, and suggestions for future research.

\section{Conclusion}
This section summarizes the main findings and conclusions of the study.

\bibliographystyle{apalike}
\bibliography{mybibliography}

\end{document}


\begin{tabular}{|c|c|c|c|c|c|}
\hline
 &  &  &  &  &  \\
\hline
 &  &  &  &  &  \\
\hline
 &  &  &  &  &  \\
\hline
 &  &  &  &  &  \\
\hline
 &  &  &  &  &  \\
\hline
 &  &  &  &  &  \\
\hline
\end{tabular}
